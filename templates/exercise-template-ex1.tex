\documentclass{article}
\usepackage{xcolor}
\usepackage{amsmath}
\usepackage{mdframed}
\usepackage{listings}
\usepackage{graphicx}
\usepackage{hyperref}
\definecolor{codegreen}{rgb}{0,0.6,0}
\definecolor{codegray}{rgb}{0.5,0.5,0.5}
\definecolor{codepurple}{rgb}{0.58,0,0.82}
\definecolor{backcolour}{rgb}{0.95,0.95,0.92}
\lstdefinestyle{mystyle}{
    backgroundcolor=\color{backcolour},   
    commentstyle=\color{codegreen},
    keywordstyle=\color{magenta},
    numberstyle=\tiny\color{codegray},
    stringstyle=\color{codepurple},
    basicstyle=\ttfamily\footnotesize,
    breakatwhitespace=false,         
    breaklines=true,                 
    captionpos=b,                    
    keepspaces=true,                 
    numbers=left,                    
    numbersep=5pt,                  
    showspaces=false,                
    showstringspaces=false,
    showtabs=false,                  
    tabsize=2
}
\lstset{style=mystyle}
\title{Exercise Ex1}
\begin{document}
\maketitle\maketitle
\begin{center}\logo\end{center}


Student:  Firstname Lastname    Sciper: 000000

\begin{mdframed}
\textbf{Please use this template to submit your answers.}\\
If you had to modify code from the notebook, please include the modified code in your submission either as screenshot or in a

\begin{verbatim}
\begin{lstlisting}[language=Python]
\end{lstlisting}
\end{verbatim}

environment.

You only need to include the code cells that you modified.

Note, that references to other parts of the documents aren't resolved in this template and will show as \texttt{??}. Check the text of the exercises on website for the reference
\end{mdframed}

\begin{mdframed}
\textbf{Exercise 1}\\
Calculate the vector product $\mathbf{c}=\mathbf{a}\times\mathbf{b}$
with

\begin{equation}
\begin{aligned}
     \mathbf{a} = \left(\begin{matrix}
     2 \\ 6 \\ 4
     \end{matrix}\right)
     \quad
     \mathbf{b} = \left(\begin{matrix}
     5 \\ 1 \\ 7
     \end{matrix}\right)
     \end{aligned}
\end{equation}

and, for the same $\mathbf{a},\mathbf{b}$, the scalar product

\begin{equation}
\begin{aligned}
     d = \mathbf{a}\cdot \mathbf{b}
     \end{aligned}
\end{equation}
\end{mdframed}

Your answer here

\begin{mdframed}
\textbf{Exercise 2}\\
Evaluate the matrix product $\mathbf{C} = \mathbf{A}\mathbf{B}$.

\begin{equation}
\mathbf{A} =\begin{aligned}
     \left(\begin{matrix}
     6 & 8 &2 \\ 9 & 1 & 5 \\ 7 & 4 & 3
     \end{matrix}\right),\quad\mathbf{B}=
      \left(\begin{matrix}
     9 & 6 & 7 \\ 5 & 4 & 4 \\  3 & 2 & 8
     \end{matrix}\right).
     \end{aligned}
\end{equation}
\end{mdframed}

Your answer here

\begin{mdframed}
\textbf{Exercise 3}\\
Evaluate the determinant for the matrix $\mathbf{A}$.

\begin{equation}
\mathbf{A}=\begin{aligned}
     \left(\begin{matrix}
     1 & 1 & 2 \\ 1 & 0 & 2 \\ 2 & 2 & 3
     \end{matrix}\right)
     \end{aligned}
\end{equation}
\end{mdframed}

Your answer here

\begin{mdframed}
\textbf{Exercise 4}\\
Does the exponent of an operator always satisfy the relation
$e^{\hat{A}+\hat{B}} = e^{\hat{A}}e^{\hat{B}}$ ? Start from the
definition of the matrix exponential.
\end{mdframed}

Your answer here

\begin{mdframed}
\textbf{Exercise 5}\\
Find the eigenvalues and eigenvectors of the matrix $\mathbf{A}$.

\begin{equation}
\mathbf{A} =\begin{aligned}
     \left(\begin{matrix}
     2 & 1 \\ -1 & 4
     \end{matrix}\right)
    \end{aligned}
\end{equation}
\end{mdframed}

Your answer here

\begin{mdframed}
\textbf{Exercise 6}\\
Show that if the product $\mathbf{C}=\mathbf{A}\mathbf{B}$ of two
Hermitian matrices is also Hermitian, then $\mathbf{A}$ and
$\mathbf{B}$ commute.
\end{mdframed}

Your answer here

\begin{mdframed}
\textbf{Exercise 7}\\
Explain the connection between the Heisenberg uncertainty principle and
the commutation relation.
\end{mdframed}

Your answer here

\begin{mdframed}
\textbf{Exercise 8}\\
What is the meaning of a multiplication of a bra and a ket

\begin{equation}
\begin{aligned}
     \left<a\middle|b\right>,
     \end{aligned}
\end{equation}

and, conversely, an operator formed by a ket and a bra?

\begin{equation}
\begin{aligned}
     \hat{\mathrm{O}} = \left|a\right>\left<b\right|
     \end{aligned}
\end{equation}
\end{mdframed}

Your answer here

\begin{mdframed}
\textbf{Exercise 9}\\
Given a basis $\left\{\psi\right\}$ for which

\begin{equation}
\begin{aligned}
     \left<\psi_i\middle|\psi_j\right> = \delta_{ij},
     \end{aligned}
\end{equation}

where $\delta_{ij}$ is the Kronecker delta, for any  state $\Psi$

\begin{equation}
\begin{aligned}
     \Ket{\Psi} = \sum_j c_j\Ket{\psi_j},
     \end{aligned}
\end{equation}

the inner product is defined as

\begin{equation}
\begin{aligned}
     \left<\Psi\middle|\Psi\right> = 1.
     \end{aligned}
\end{equation}

Show that this holds only as long as

\begin{equation}
\begin{aligned}
     \sum_j c_j^2 = 1,
     \end{aligned}
\end{equation}

where the $c_j$ are the aforementioned expansion coefficients.
\end{mdframed}

Your answer here

\begin{mdframed}
\textbf{Bonus Exercise 10}\\
Prove that, given the above conditions,
$c_j=\left<\psi_j\middle|\Psi\right>$.
\end{mdframed}

Your answer here

\begin{mdframed}
\textbf{Exercise 11}\\
Diagonalise the matrices $\mathbf{A}$ and $\mathbf{B}$. Specify which one is Hermitian.

\begin{equation}
\mathbf{A}=\begin{aligned}
     \left(\begin{matrix}
     1 & 1-i \\ 1+i & 2
     \end{matrix}\right)
     , \quad
     \mathbf{B} =\left(\begin{matrix}
     1 & 1 \\ 0 & 1
     \end{matrix}\right)
     \end{aligned}
\end{equation}
\end{mdframed}

Your answer here

\begin{mdframed}
\textbf{Bonus Exercise 12}\\
Prove that the eigenvalues of a Hermitian operator are
real.
\end{mdframed}

Your answer here

\begin{mdframed}
\textbf{Exercise 13}\\
Give the position and the momentum operators (consider only one
dimension) in the position representation.
\end{mdframed}

Your answer here

\begin{mdframed}
\textbf{Exercise 14}\\
Give the commutator of the position and linear momentum operators in
the position representation (consider one dimension only).
\end{mdframed}

Your answer here

\begin{mdframed}
\textbf{Exercise 15}\\
Is the electronic Hamiltonian $\hat{H}_{el}$ a linear operator and
why (not)?
\end{mdframed}

Your answer here

\begin{mdframed}
\textbf{Exercise 16}\\
Show that, if two operators $\hat{\mathrm{A}}$, $\hat{\mathrm{B}}$
commute and if $\Ket{\psi}$ is an eigenvector of $\hat{\mathrm{A}}$,
$\hat{\mathrm{B}}\Ket{\psi}$ is an eigenvector of
$\hat{\mathrm{A}}$, too, with the same eigenvalue.

\textbf{Bonus}:  If $\Ket{\psi}$ is part of a set of degenerate
eigenvectors, show that the subspace spanned by the eigenvalues of
$\hat{\mathrm{A}}$ is invariant under the action of
$\hat{\mathrm{B}}$.
\end{mdframed}

Your answer here

\begin{mdframed}
\textbf{Exercise 17}\\
Demonstrate that, if two hermitian operators $\hat{\mathrm{A}}$,
$\hat{\mathrm{B}}$ commute and ${\Ket{\psi_1}}$, ${\Ket{\psi_2}}$
are eigenvectors of $\hat{\mathrm{A}}$ associated to different
eigenvalues, then the matrix element
$\Bra{\psi_1}\hat{\mathrm{B}}\Ket{\psi_2}$ vanishes.
\end{mdframed}

Your answer here

\begin{mdframed}
\textbf{Bonus Exercise 18}\\
Show that the potential energy operator
$\hat{\mathrm{V}}(\mathbf{r})$ is multiplicative when applied to the
real-space wavefunction.
\end{mdframed}

Your answer here

\begin{mdframed}
\textbf{Bonus Exercise 19}\\
The link between position and momentum representation is
given by a Fourier transform. Explain how this relates to the Heisenberg
uncertainty principle.
\end{mdframed}

Your answer here

\begin{mdframed}
\textbf{Exercise 20}\\
In a system that consists of only two states (such as an electron spin
in a magnetic field, where the electron spin can be in one of two
orientations), the Hamiltonian has the following matrix elements:
$H_{11}=a, \ H_{22}=b, \ H_{12}=d, \ H_{21}=d$. How would you determine
the energy levels $E$ and the eigenstates $\mathbf{\Psi}$ of the system?
(You do not need to solve this problem explicitly, merely outlining the
procedure is sufficient.)
\end{mdframed}

Your answer here

\begin{mdframed}
\textbf{Exercise 21}\\
Define two vectors, \texttt{$\phi$1} and \texttt{$\phi$2}, with two elements each, that are normalized, in the sense $\langle\phi_i|\phi_i\rangle=1$, and orthogonal in the sense that $\langle\phi_i|\phi_j\rangle=0$.

\textbf{Hint:} In \texttt{numpy} a vector \texttt{v} with the two elements \texttt{1} and \texttt{2} is defined through the command

\begin{verbatim}
v=np.array([1,2])
\end{verbatim}
\end{mdframed}

Your answer here

\begin{mdframed}
\textbf{Exercise 22}\\
Show that $\phi_1$ and $\phi_2$ are normalized and orthonormal

\textbf{Hint:} Here are reported some useful \texttt{numpy} functions to work with vectors:

\begin{itemize}
\item \texttt{v.dot(w)} - inner product (scalar product) of two vectors \texttt{v}, \texttt{w}
\item \texttt{v.conj()} - complex conjugate of a vector \texttt{v}
\item \texttt{v.conj().dot(w)} - inner product of $v^\dagger w$
\end{itemize}
\end{mdframed}

Your answer here


\end{document}
