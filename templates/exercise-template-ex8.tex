\documentclass{article}
\usepackage{xcolor}
\usepackage{mdframed}
\usepackage{listings}
\usepackage{graphicx}
\usepackage{hyperref}
\definecolor{codegreen}{rgb}{0,0.6,0}
\definecolor{codegray}{rgb}{0.5,0.5,0.5}
\definecolor{codepurple}{rgb}{0.58,0,0.82}
\definecolor{backcolour}{rgb}{0.95,0.95,0.92}
\lstdefinestyle{mystyle}{
    backgroundcolor=\color{backcolour},   
    commentstyle=\color{codegreen},
    keywordstyle=\color{magenta},
    numberstyle=\tiny\color{codegray},
    stringstyle=\color{codepurple},
    basicstyle=\ttfamily\footnotesize,
    breakatwhitespace=false,         
    breaklines=true,                 
    captionpos=b,                    
    keepspaces=true,                 
    numbers=left,                    
    numbersep=5pt,                  
    showspaces=false,                
    showstringspaces=false,
    showtabs=false,                  
    tabsize=2
}
\lstset{style=mystyle}
\title{Exercise Ex8}
\begin{document}
\maketitle\maketitle

Student:  Firstname Lastname    Sciper: 000000

\begin{mdframed}
\textbf{Please use this template to submit your answers.}\\
If you had to modify code from the notebook, please include the modified code in your submission either as screenshot or in a

\begin{verbatim}
\begin{lstlisting}[language=Python]
\end{lstlisting}
\end{verbatim}

environment.

You only need to include the code cells that you modified.

Note, that references to other parts of the documents aren't resolved in this template and will show as \texttt{??}. Check the text of the exercises on website for the reference
\end{mdframed}

\begin{mdframed}
\textbf{Exercise 2}\\
Does the bond length of the $H_3CH_2C -CH_2CH_3$ bond fluctuate as
the dihedral is changed? Why (not)?
\end{mdframed}

Your answer here

\begin{mdframed}
\textbf{Exercise 3}\\
Create a potential energy profile along the dihedral by plotting the
energies of the system as a function of the scanned dihedral angle as done above.
Then, to each maximum and minimum, assign the corresponding
conformation from your trajectory. Show these conformations in your
report.
Assign also names to the conformations according to the \textit{Newman
projections} you know from organic chemistry (where possible).
\end{mdframed}

Your answer here

\begin{mdframed}
\textbf{Exercise 4}\\
The rotational barrier of butane lies between 5 and 6 kcal
mol\textsuperscript{-1} [\^Murcko1996]. Compare this value with the one you calculated and explain possible sources of error, leading to higher or lower barriers.
\end{mdframed}

Your answer here
%%%%%%%%%%%%%%%%%%%%%%%%%%%%%%%%%%%%%%%%%%%%%%%%%%
%%%%%%%%%%%%%%  acronyms & glossary  %%%%%%%%%%%%%

%%%%%%%%%%%%%%%%%%%%%%%%%%%%%%%%%%%%%%%%%%%%%%%%%%


\clearpage


\end{document}
