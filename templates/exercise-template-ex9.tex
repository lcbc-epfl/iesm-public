\documentclass{article}
\usepackage{xcolor}
\usepackage{amsmath}
\usepackage{mdframed}
\usepackage{listings}
\usepackage{graphicx}
\usepackage{hyperref}
\definecolor{codegreen}{rgb}{0,0.6,0}
\definecolor{codegray}{rgb}{0.5,0.5,0.5}
\definecolor{codepurple}{rgb}{0.58,0,0.82}
\definecolor{backcolour}{rgb}{0.95,0.95,0.92}
\lstdefinestyle{mystyle}{
    backgroundcolor=\color{backcolour},   
    commentstyle=\color{codegreen},
    keywordstyle=\color{magenta},
    numberstyle=\tiny\color{codegray},
    stringstyle=\color{codepurple},
    basicstyle=\ttfamily\footnotesize,
    breakatwhitespace=false,         
    breaklines=true,                 
    captionpos=b,                    
    keepspaces=true,                 
    numbers=left,                    
    numbersep=5pt,                  
    showspaces=false,                
    showstringspaces=false,
    showtabs=false,                  
    tabsize=2
}
\lstset{style=mystyle}
\title{Exercise Ex9}
\begin{document}
\maketitle\maketitle

Student:  Firstname Lastname    Sciper: 000000

\begin{mdframed}
\textbf{Please use this template to submit your answers.}\\
If you had to modify code from the notebook, please include the modified code in your submission either as screenshot or in a

\begin{verbatim}
\begin{lstlisting}[language=Python]
\end{lstlisting}
\end{verbatim}

environment.

You only need to include the code cells that you modified.

Note, that references to other parts of the documents aren't resolved in this template and will show as \texttt{??}. Check the text of the exercises on website for the reference
\end{mdframed}

\begin{mdframed}
\textbf{Exercise 1}\\
What kind of reaction mechanism would you expect? Would you expect
the stereochemistry at the chiral carbon to be preserved? Identify
the chirality centre in the chloropropanoate and the product
epoxide.
\end{mdframed}

Your answer here

\begin{mdframed}
\textbf{Exercise 2}\\
Suggest possible transition state structures.
\end{mdframed}

Your answer here

\begin{mdframed}
\textbf{Exercise 3}\\
Include a screenshot of the transition state in your report that you obtained and the corresponding Z-matrix for the transition state guess
\end{mdframed}

Your answer here

\begin{mdframed}
\textbf{Exercise 4}\\
Take a screenshot of the optimised transition state structure. How did the structure change with respect to the constrained-optimised guess?
\end{mdframed}

Your answer here

\begin{mdframed}
\textbf{Exercise 4}\\
Report the value of the negative frequency that you obtain.  What motion is this mode related to?
What motions are associated with low and high vibrational frequencies? Choose two positive vibrational modes and desribe their particular associated motion.
\end{mdframed}

Your answer here

\begin{mdframed}
\textbf{Exercise 5}\\
Is the transition state you predicted an early or a late transition state? What about the guess?
\end{mdframed}

Your answer here

\begin{mdframed}
\textbf{Exercise 6}\\
Having found a transition state, how would you now obtain the barrier height for your reaction? Are there ways of verifying whether you have found a meaningful
transition state? \textbf{Bonus:} How would you define `meaningful' in
this context?
\end{mdframed}

Your answer here

\begin{mdframed}
\textbf{Exercise 8}\\
Is the stereochemistry at the carbon at which the reaction takes
place retained?
\end{mdframed}

Your answer here

\begin{mdframed}
\textbf{Exercise 7}\\
Take a screenshot of the graph of the potential energy profile you
recorded.
Why is the barrier for the epoxide formation so low? Will this be
the overall barrier for the reaction as depicted in the previous section?
\end{mdframed}

Your answer here

\begin{mdframed}
\textbf{Exercise  9}\\
How do the C-Cl and the two relevant C-O bond lengths change during
the trajectory? Does the C-C bond in the ring contract as the
epoxide is formed? Show a graph depicting the evolution of these
parameters as the reaction progresses.
\end{mdframed}

Your answer here

\begin{mdframed}
\textbf{Exercise 10}\\
What is happening to the methyl group as the reaction proceeds? Find
a suitable parameter (angle, dihedral) to describe and characterise
possible changes you observe, change the code below. Explain in your report what atoms you considered and take and include the evolutions of the chosen parameters during the IRC procedure.
\end{mdframed}

Your answer here
%%%%%%%%%%%%%%%%%%%%%%%%%%%%%%%%%%%%%%%%%%%%%%%%%%
%%%%%%%%%%%%%%  acronyms & glossary  %%%%%%%%%%%%%

%%%%%%%%%%%%%%%%%%%%%%%%%%%%%%%%%%%%%%%%%%%%%%%%%%

\end{document}
