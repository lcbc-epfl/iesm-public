\documentclass{article}
\usepackage{xcolor}
\usepackage{amsmath}
\usepackage{mdframed}
\usepackage{listings}
\usepackage{graphicx}
\usepackage{hyperref}
\definecolor{codegreen}{rgb}{0,0.6,0}
\definecolor{codegray}{rgb}{0.5,0.5,0.5}
\definecolor{codepurple}{rgb}{0.58,0,0.82}
\definecolor{backcolour}{rgb}{0.95,0.95,0.92}
\lstdefinestyle{mystyle}{
    backgroundcolor=\color{backcolour},   
    commentstyle=\color{codegreen},
    keywordstyle=\color{magenta},
    numberstyle=\tiny\color{codegray},
    stringstyle=\color{codepurple},
    basicstyle=\ttfamily\footnotesize,
    breakatwhitespace=false,         
    breaklines=true,                 
    captionpos=b,                    
    keepspaces=true,                 
    numbers=left,                    
    numbersep=5pt,                  
    showspaces=false,                
    showstringspaces=false,
    showtabs=false,                  
    tabsize=2
}
\lstset{style=mystyle}
\title{Exercise Ex4}
\begin{document}
\maketitle\maketitle
\begin{center}\logo\end{center}


Student:  Firstname Lastname    Sciper: 000000

\begin{mdframed}
\textbf{Please use this template to submit your answers.}\\
If you had to modify code from the notebook, please include the modified code in your submission either as screenshot or in a

\begin{verbatim}
\begin{lstlisting}[language=Python]
\end{lstlisting}
\end{verbatim}

environment.

You only need to include the code cells that you modified.

Note, that references to other parts of the documents aren't resolved in this template and will show as \texttt{??}. Check the text of the exercises on website for the reference
\end{mdframed}

\begin{mdframed}
\textbf{Exercise 1}\\
Define $S_{ij}$ using Dirac notation.
\end{mdframed}

Your answer here

\begin{mdframed}
\textbf{Exercise 2}\\
For an orthonormal basis, what does the overlap integral array, \texttt{S}, look like?
\end{mdframed}

Your answer here

\begin{mdframed}
\textbf{Exercise 3}\\
Use \texttt{B} and \texttt{B\_dagger} and the matrix rules above to calculate the matrix \texttt{S}.
\end{mdframed}

Your answer here

\begin{mdframed}
\textbf{Exercise 4}\\
Describe how the notation of the \texttt{np.einsum} command correlates to the implicit summation formula written above.
\end{mdframed}

Your answer here

\begin{mdframed}
\textbf{Exercise 5}\\
Use the function \texttt{np.einsum()} to calculate the matrix \texttt{S}, and confirm that your answer is the same as above.
\end{mdframed}

Your answer here

\begin{mdframed}
\textbf{Exercise 6}\\
Propose a different orthonormal basis, modify \texttt{phi1} and \texttt{phi2}, and verify that \texttt{S} still has the same form. There are infinitely many choices. It isn't complex... or \textit{is} it?!
\end{mdframed}

Your answer here

\begin{mdframed}
\textbf{Exercise 7}\\
How many electrons are there in total in H$_2$O?
How many occupied molecular orbitals would you expect?
\end{mdframed}

Your answer here

\begin{mdframed}
\textbf{Exercise 8}\\
Explain the shape (number of rows and columns) of \texttt{S} in terms of the AO basis set we chose.
\end{mdframed}

Your answer here

\begin{mdframed}
\textbf{Exercise 9}\\
Based on your observations of \texttt{S} in the AO basis, answer the following questions

\begin{enumerate}
\item What do the diagonal elements of \texttt{S} indicate?
\item What do the off-diagonal elements of \texttt{S} indicate?
\item Does the Gaussian atomic orbital basis set form an orthonormal basis?
\end{enumerate}
\end{mdframed}

Your answer here

\begin{mdframed}
\textbf{Exercise 10}\\
Does the result of your extra evaluation agree with what you determined previously?
\end{mdframed}

Your answer here

\begin{mdframed}
\textbf{Exercise 11}\\
Use the function \texttt{np.linalg.inv()} to calculate the inverse of \texttt{S}, and the function \texttt{splinalg.sqrtm()} to take its (matrix) square root. Execute the code below and examine the matrix \texttt{A}.
\end{mdframed}

Your answer here

\begin{mdframed}
\textbf{Exercise 12}\\
What do you observe about the elements of \texttt{A}? Is the matrix real or complex? Is the matrix symmetric or not?
\end{mdframed}

Your answer here

\begin{mdframed}
\textbf{Exercise 13}\\
Use the orthogonalization matrix \texttt{A} to transform the overlap matrix, \texttt{S}. Check the transformed overlap matrix, \texttt{S\_p}, to make sure it represents an orthonormal basis.
\end{mdframed}

Your answer here

\begin{mdframed}
\textbf{Exercise 14}\\
The product A S A does not take the complex conjugate transpose of A. What conditions (properties of A) make that ok?
\end{mdframed}

Your answer here

\begin{mdframed}
\textbf{Exercise 15}\\
Based on the definition of $C'$, propose a definition of $C$ in terms of $A$ and $C'$. Justify your equation.
\end{mdframed}

Your answer here

\begin{mdframed}
\textbf{Exercise 16}\\
In the cell below, use the core Hamiltonian matrix as your initial guess for the Fock matrix. Transform it with the same A matrix you used above.  To calculate the eigenvalues, \texttt{vals}, and eigenvectors, \texttt{vecs}, of matrix \texttt{M} using  \texttt{vals, vecs = np.linalg.eigh(M)}.
\end{mdframed}

Your answer here

\begin{mdframed}
\textbf{Exercise 17}\\
Display, i.e., \texttt{print}, the coefficent matrix and confirm it the correct size
\end{mdframed}

Your answer here

\begin{mdframed}
\textbf{Exercise 18}\\
Use \texttt{A} and the formula you proposed previously to transform the coefficient matrix back to the AO basis. Confirm that the resulting matrix appears reasonable, i.e., similar size and magnitude
\end{mdframed}

Your answer here

\begin{mdframed}
\textbf{Exercise 19}\\
Build the density matrix, \texttt{D}, from the occupied orbitals, \texttt{C\_occ}, using the function \texttt{np.einsum()}. \textbf{Hint} Look at (\ref{DensityMatrix})
\end{mdframed}

Your answer here

\begin{mdframed}
\textbf{Exercise 20}\\
Define J  in terms of the density matrix, \texttt{D}, and the electron repulsion integral tensor, \texttt{I}, using \texttt{np.einsum()}.
\end{mdframed}

Your answer here

\begin{mdframed}
\textbf{Exercise 21}\\
Define K  in terms of the density matrix, \texttt{D}, and the electron repulsion integral tensor, \texttt{I}, using \texttt{einsum()}.
\end{mdframed}

Your answer here

\begin{mdframed}
\textbf{Exercise 22}\\
Define F in terms of H, J, and K. (Recall The~Hartree-Fock~procedure)
\end{mdframed}

Your answer here

\begin{mdframed}
\textbf{Exercise 23}\\
Calculate the SCF energy based on H, F, and D using \texttt{np.einsum()}.
\end{mdframed}

Your answer here

\begin{mdframed}
\textbf{Exercise 24}\\
Based on the result of the calculation in \ref{basisset}, is this a reasonable answer?
\end{mdframed}

Your answer here

\begin{mdframed}
\textbf{Exercise 25}\\
Describe a procedure (i.e. identify the steps) that will update coefficients and compute a new density matrix based on the updated values of the Fock matrix.
\end{mdframed}

Your answer here

\begin{mdframed}
\textbf{Exercise 26}\\
Using the procedure proposed above, calculate the updated coefficients
\end{mdframed}

Your answer here

\begin{mdframed}
\textbf{Bonus Exercise 27}\\
Modify the value of E\_conv and describe its effect the number of iterations.
\end{mdframed}

Your answer here


\end{document}
