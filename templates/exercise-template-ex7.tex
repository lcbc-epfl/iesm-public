\documentclass{article}
\usepackage{xcolor}
\usepackage{mdframed}
\usepackage{listings}
\usepackage{graphicx}
\usepackage{hyperref}
\definecolor{codegreen}{rgb}{0,0.6,0}
\definecolor{codegray}{rgb}{0.5,0.5,0.5}
\definecolor{codepurple}{rgb}{0.58,0,0.82}
\definecolor{backcolour}{rgb}{0.95,0.95,0.92}
\lstdefinestyle{mystyle}{
    backgroundcolor=\color{backcolour},   
    commentstyle=\color{codegreen},
    keywordstyle=\color{magenta},
    numberstyle=\tiny\color{codegray},
    stringstyle=\color{codepurple},
    basicstyle=\ttfamily\footnotesize,
    breakatwhitespace=false,         
    breaklines=true,                 
    captionpos=b,                    
    keepspaces=true,                 
    numbers=left,                    
    numbersep=5pt,                  
    showspaces=false,                
    showstringspaces=false,
    showtabs=false,                  
    tabsize=2
}
\lstset{style=mystyle}
\title{Exercise Ex7}
\begin{document}
\maketitle\maketitle

Student:  Firstname Lastname    Sciper: 000000

\begin{mdframed}
\textbf{Please use this template to submit your answers.}\\
If you had to modify code from the notebook, please include the modified code in your submission either as screenshot or in a

\begin{verbatim}
\begin{lstlisting}[language=Python]
\end{lstlisting}
\end{verbatim}

environment.

You only need to include the code cells that you modified.

Note, that references to other parts of the documents aren't resolved in this template and will show as \texttt{??}. Check the text of the exercises on website for the reference
\end{mdframed}

\begin{mdframed}
\textbf{Exercise 1}\\
Fix the error in the single-point wavefunction optimization for the allyl radical using the BLYP exchange-correlation functional.
\end{mdframed}

Your answer here

\begin{mdframed}
\textbf{Exercise 2}\\
Fix the error in the single-point wavefunction optimisation of the $Cu[H_{2}O]^{2+}$ aqua complex (as observed in the gas phase) using the PBE exchange-correlation functional.
\end{mdframed}

Your answer here

\begin{mdframed}
\textbf{Exercise 3}\\
Fix the error in the geometry optimisation of $N_2$ using the BP86 exchange-correlation functional.
\end{mdframed}

Your answer here

\begin{mdframed}
\textbf{Exercise 4}\\
Fix the error in the single point calculation of  $^3 O$ using the B97-2 exchange-correlation functional.
\end{mdframed}

Your answer here

\begin{mdframed}
\textbf{Exercise 5}\\
Before running the calculations, give your best estimate as to which of the reaction energies (i.e. the energy difference between product and reactants) do you think is most ``difficult'' to calculate? Rank them in order of increasing difficulty and explain the reasoning behind your choices.

Difficult in this case refers to getting an accurate reaction energy with low absolute error to true energy (reference energy from the dataset). As chemical accuracy we usually use 1 kcal mol\textsuperscript{-1}
\end{mdframed}

Your answer here

\begin{mdframed}
\textbf{Exercise 6}\\
Run the calculations below and check whether your assessment of the hard, medium and easy reactions is correct. Include the tables containing the calculated energies for the three problems in your report. Which functional do you think contained the reaction in its training data?
\end{mdframed}

Your answer here

\begin{mdframed}
\textbf{Exercise 8}\\
What is the impact of using the dispersion correction for the ethane complex? For what phenomena are such corrections relevant? Give a real world example system.
\end{mdframed}

Your answer here

\begin{mdframed}
\textbf{Exercise 9}\\
Which grid size should you use for this kind of problem? Are the fluctuations observed with the SG1 grid physical? Evaluate plots you create above.
\end{mdframed}

Your answer here

\begin{mdframed}
\textbf{Exercise 10}\\
Compare the two functionals regarding convergence with respect to the integration grid. Which functional shows concerning behavior in this regard? Evaluate plots you create below
\end{mdframed}

Your answer here
%%%%%%%%%%%%%%%%%%%%%%%%%%%%%%%%%%%%%%%%%%%%%%%%%%
%%%%%%%%%%%%%%  acronyms & glossary  %%%%%%%%%%%%%

%%%%%%%%%%%%%%%%%%%%%%%%%%%%%%%%%%%%%%%%%%%%%%%%%%

\end{document}
