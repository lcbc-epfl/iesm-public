\documentclass{article}
\usepackage{xcolor}
\usepackage{amsmath}
\usepackage{mdframed}
\usepackage{listings}
\usepackage{graphicx}
\usepackage{hyperref}
\definecolor{codegreen}{rgb}{0,0.6,0}
\definecolor{codegray}{rgb}{0.5,0.5,0.5}
\definecolor{codepurple}{rgb}{0.58,0,0.82}
\definecolor{backcolour}{rgb}{0.95,0.95,0.92}
\lstdefinestyle{mystyle}{
    backgroundcolor=\color{backcolour},   
    commentstyle=\color{codegreen},
    keywordstyle=\color{magenta},
    numberstyle=\tiny\color{codegray},
    stringstyle=\color{codepurple},
    basicstyle=\ttfamily\footnotesize,
    breakatwhitespace=false,         
    breaklines=true,                 
    captionpos=b,                    
    keepspaces=true,                 
    numbers=left,                    
    numbersep=5pt,                  
    showspaces=false,                
    showstringspaces=false,
    showtabs=false,                  
    tabsize=2
}
\lstset{style=mystyle}
\title{Exercise Ex3}
\begin{document}
\maketitle\maketitle

Student:  Firstname Lastname    Sciper: 000000

\begin{mdframed}
\textbf{Please use this template to submit your answers.}\\
If you had to modify code from the notebook, please include the modified code in your submission either as screenshot or in a

\begin{verbatim}
\begin{lstlisting}[language=Python]
\end{lstlisting}
\end{verbatim}

environment.

You only need to include the code cells that you modified.

Note, that references to other parts of the documents aren't resolved in this template and will show as \texttt{??}. Check the text of the exercises on website for the reference
\end{mdframed}

\begin{mdframed}
\textbf{Exercise 1}\\
Include a table of the the calculated energies using the three different basis sets
\end{mdframed}

Your answer here

\begin{mdframed}
\textbf{Exercise 2}\\
What is the meaning of the Delta E column? What is DIIS? (Hint: to get more details, it may help to check the \href{https://psicode.org/psi4manual/master/scf.html}{HF page on psi4 manual})
\end{mdframed}

Your answer here

\begin{mdframed}
\textbf{Exercise 3}\\
How does the basis set size influence the SCF convergence behaviour? Include in your report the plot of the energy convergence along SCF iterations.
\end{mdframed}

Your answer here

\begin{mdframed}
\textbf{Exercise 4}\\
Explain the physical origin of the difference between the two
dissociation curves. Discuss how important this difference is by
comparing to the energy that is normally required to break a bond.
\end{mdframed}

Your answer here

\begin{mdframed}
\textbf{Exercise 5 - Bonus}\\
Try to find out why we are using \texttt{guess\_mix:True} and \texttt{'guess':'gwh'}. \textbf{Hint}: What happend if you use the same settings as for the RHF calculation?

You can also look at the Psi4 manual to find out about the different options for the initial guess.
\end{mdframed}

Your answer here

\begin{mdframed}
\textbf{Exercise 6}\\
Why are there multiple iterations of the SCF cycle?
\end{mdframed}

Your answer here

\begin{mdframed}
\textbf{Exercise 7}\\
Why does the optimization finish after 8 steps?
\end{mdframed}

Your answer here

\begin{mdframed}
\textbf{Exercise 8}\\
Try to start from two (or more!) different suboptimal geometries of water and include the plots of the potential energy surface in the report, together with a screenshot of the starting and final geometry and/or the values of the O-H bond and H-O-H angle.

What do the points on the orange curve correspond to? Connect them to the flowchart at the beginning of this exercise?
\end{mdframed}

Your answer here
%%%%%%%%%%%%%%%%%%%%%%%%%%%%%%%%%%%%%%%%%%%%%%%%%%
%%%%%%%%%%%%%%  acronyms & glossary  %%%%%%%%%%%%%

%%%%%%%%%%%%%%%%%%%%%%%%%%%%%%%%%%%%%%%%%%%%%%%%%%

\end{document}
